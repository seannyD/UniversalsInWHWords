\documentclass[8pt,landscape,a4paper]{article}
\usepackage[margin=0.5in]{geometry} % see geometry.pdf on how to lay out the page. There's lots.
\usepackage{natbib}
%\usepackage{citeref}
\usepackage{graphicx}
\usepackage{color}
\usepackage{url}
\usepackage{pdflscape}
\usepackage[table]{xcolor}
\usepackage{longtable}

\geometry{a4paper} % or letter or a5paper or ... etc
% \geometry{landscape} % rotated page geometry
\usepackage{setspace}
\title{A case for systematic sound symbolism in pragmatics: Universals in wh-words \\ List of languages used}
\author{}
\date{} % delete this line to display the current date

%%% BEGIN DOCUMENT
\begin{document}
\maketitle

This is a list of languages used in the main analyses.  It shows:
\begin{itemize}
\item the name of the language
\item the Glottolog and ISO identifier
\item the language family and area
\item the source of the word list
\item the position of interrogative phrases in the language
\item the source for the interrogative position data (WALS refers to \cite{wals-93})
\item the reference for the interrogative position data if not in WALS
\end{itemize}

\subsection*{Coding interrogative position}
34 languages were coded by two coders (one of the authors, AS and and a naive coder, JP) according to the typology in \cite{wals-93}.  They agreed on 28 out of 34 languages (82\%).  A third coder (one of the authors, SR) blindly coded 6 random languages and agreed on all with both of the other coders (and cited the same grammars and page numbers as evidence).  The third coder then coded languages that the first two coders disagreed on (without knowing the coding decisions), and made an executive decision.  5 of the 6 decisions sided with AS. 

\clearpage
\newpage

{\scriptsize
% latex table generated in R 3.3.1 by xtable 1.8-2 package
% Thu Mar 30 17:12:29 2017
%\begin{table}[ht]
\centering
\begin{longtable}{lllllllll}
  \hline
Name & Glottocode & WALS code & Family & Area & Source & Interrogative & IP Source & Reference \\ 
  \hline
Polci & polc1243 &  & Afro-Asiatic & African Savannah & IDS &  &  &  \\ 
  Gawwada & gaww1239 & dug & Afro-Asiatic & Greater Abyssinia & WOLD &  &  &  \\ 
  Hausa & haus1257 & hau & Afro-Asiatic & African Savannah & WOLD & Mixed & WALS &  \\ 
  Tarifiyt Berber & tari1263 & brf & Afro-Asiatic & N Africa & WOLD & Initial & WALS &  \\ 
  Iraqw & iraq1241 & irq & Afro-Asiatic & S Africa & WOLD & Non-Initial & WALS &  \\ 
  Mapudungun & mapu1245 & map & Araucanian & Andean & WOLD & Non-Initial & WALS &  \\ 
  Mashco Piro & mash1270 &  & Arawakan & NE South America & IDS &  &  &  \\ 
  Wapishana & wapi1253 & wps & Arawakan & NE South America & IDS &  &  &  \\ 
  Waurá & waur1244 & wur & Arawakan & NE South America & IDS &  &  &  \\ 
  Trinitario & trin1274 &  & Arawakan & NE South America & IDS & Initial & S\&R & \cite[91]{rose_trinitario2014.pdf} \\ 
  Yavitero & yavi1244 &  & Arawakan & NE South America & IDS & Initial & S\&R & \cite[638]{mosonyietal_yavitero2000.pdf}\cite[79...]{mosonyi_yavitero1987.pdf} \\ 
  Goajiro & wayu1243 & goa & Arawakan & NE South America & IDS & Initial & WALS &  \\ 
  Ignaciano & igna1246 & ign & Arawakan & NE South America & IDS & Initial & WALS &  \\ 
  Swahili & swah1253 & swa & Atlantic-Congo & S Africa & WOLD & Non-Initial & WALS &  \\ 
  Ahlao Th & aheu1239 &  & Austroasiatic & Southeast Asia & IDS &  &  &  \\ 
  Bulang-2 & blan1242 &  & Austroasiatic & Southeast Asia & IDS &  &  &  \\ 
  Cuoi & cuoi1242 &  & Austroasiatic & Southeast Asia & IDS &  &  &  \\ 
  Iduh & oduu1239 &  & Austroasiatic & Southeast Asia & IDS &  &  &  \\ 
  Keme  & manm1238 &  & Austroasiatic & Southeast Asia & IDS &  &  &  \\ 
  Kha Poong & khap1242 &  & Austroasiatic & Southeast Asia & IDS &  &  &  \\ 
  Malieng & mali1278 &  & Austroasiatic & Southeast Asia & IDS &  &  &  \\ 
  Mang Ch & mang1378 &  & Austroasiatic & Southeast Asia & IDS &  &  &  \\ 
  Mlabri & mlab1235 & mlm & Austroasiatic & Southeast Asia & IDS &  &  &  \\ 
  Nyakur & nyah1250 & nkt & Austroasiatic & Southeast Asia & IDS &  &  &  \\ 
  Paliu & boly1239 &  & Austroasiatic & Southeast Asia & IDS &  &  &  \\ 
  Phong & phon1246 &  & Austroasiatic & Southeast Asia & IDS &  &  &  \\ 
  Prai & pray1239 &  & Austroasiatic & Southeast Asia & IDS &  &  &  \\ 
  Pusing & bitt1240 &  & Austroasiatic & Southeast Asia & IDS &  &  &  \\ 
  Rumai & ruma1248 &  & Austroasiatic & Southeast Asia & IDS &  &  &  \\ 
  Samre & samr1245 &  & Austroasiatic & Southeast Asia & IDS &  &  &  \\ 
  Wa & nucl1290 &  & Austroasiatic & Southeast Asia & IDS &  &  &  \\ 
  Xu & huuu1240 &  & Austroasiatic & Southeast Asia & IDS &  &  &  \\ 
  CHNG  & west2394 & ksn & Austroasiatic & Southeast Asia & IDS & Non-Initial & WALS &  \\ 
  Chong & chon1284 & ksn & Austroasiatic & Southeast Asia & IDS & Non-Initial & WALS &  \\ 
  Kasong & suoy1242 & ksn & Austroasiatic & Southeast Asia & IDS & Non-Initial & WALS &  \\ 
  Pear & pear1247 & ksn & Austroasiatic & Southeast Asia & IDS & Non-Initial & WALS &  \\ 
  Surin Khmer & suri1265 & khm & Austroasiatic & Southeast Asia & IDS & Non-Initial & WALS &  \\ 
  Khasi & khas1269 & khs & Austroasiatic & Indic & Sprakbanken & Non-Initial & WALS &  \\ 
  Ceq Wong & chew1245 &  & Austroasiatic & Southeast Asia & WOLD &  &  &  \\ 
  Vietnamese & viet1252 & vie & Austroasiatic & Southeast Asia & WOLD & Non-Initial & WALS &  \\ 
  Tuamotuan & tuam1242 & tua & Austronesian & Oceania & IDS &  &  &  \\ 
  Rapa Nui & rapa1244 & rap & Austronesian & Oceania & IDS & Initial & S\&R & \cite[19]{feu_rapanui1996}\cite[73]{weberrobert_rapanui1988_o.pdf} \\ 
  Hawaiian & hawa1245 & haw & Austronesian & Oceania & IDS & Non-Initial & WALS &  \\ 
  Maori & maor1246 & mao & Austronesian & Oceania & IDS & Non-Initial & WALS &  \\ 
  Rotuman & rotu1241 & rot & Austronesian & Oceania & IDS & Non-Initial & WALS &  \\ 
  Takia & taki1248 & tak & Austronesian & N Coast New Guinea & WOLD &  &  &  \\ 
  Indonesian & indo1316 & ind & Austronesian & Southeast Asia & WOLD & Mixed & WALS &  \\ 
  Malagasy & plat1254 & mal & Austronesian & Southeast Asia & WOLD & Initial & WALS &  \\ 
  Aymara & cent2142 & aym & Aymara & Andean & IDS & Non-Initial & WALS &  \\ 
  Colorado & colo1256 & tsf & Barbacoan & Andean & IDS &  &  &  \\ 
  Cayapa & chac1249 & cay & Barbacoan & Andean & IDS & Initial & WALS &  \\ 
  Basque & basq1248 & bqb & Basque & Europe & IDS & Non-Initial & WALS &  \\ 
  Wai Wai & waiw1244 & wai & Cariban & NE South America & IDS &  &  &  \\ 
  Panare & enap1235 & pnr & Cariban & NE South America & IDS & Initial & S\&R & \cite[380]{payne-payne_panare2013.pdf} \\ 
  De'kwana & maqu1238 & cde & Cariban & NE South America & IDS & Initial & WALS &  \\ 
  Macushi & macu1259 & mac & Cariban & NE South America & IDS & Initial & WALS &  \\ 
  Kali'na & gali1262 & car & Cariban & NE South America & WOLD &  &  &  \\ 
  Cayuvava & cayu1262 & cyv & Cayubaba & NE South America & IDS &  &  &  \\ 
  Barí & bari1297 & mti & Chibchan & NE South America & IDS &  &  &  \\ 
  Muisca & chib1270 & msc & Chibchan & NE South America & IDS & Initial & S\&R & \cite[22]{ostler_muisca1994_o.pdf} \\ 
  Embera & nort2972 & emb & Chocoan & Andean & IDS & Initial & S\&R & \cite[13]{mortensen_northern-embera1999v2} \\ 
  Epena & epen1239 & epe & Chocoan & Andean & IDS & Initial & WALS &  \\ 
  Selknam & onaa1245 & sel & Chonan & SE South America & IDS &  &  &  \\ 
  Tehuelche & tehu1242 & teh & Chonan & SE South America & IDS & Initial & S\&R & \cite[425]{garay_tehuelche1998.pdf} \\ 
  Cofán & cofa1242 & cof & Cofán & Andean & IDS &  &  &  \\ 
  Tamil & tami1289 & tml & Dravidian & Indic & IDS & Non-Initial & WALS &  \\ 
  Telugu & telu1262 & tel & Dravidian & Indic & IDS & Non-Initial & WALS &  \\ 
  Mocoví & moco1246 & mcv & Guaicuruan & SE South America & IDS & Initial & WALS &  \\ 
  Pilagá & pila1245 & pga & Guaicuruan & SE South America & IDS & Initial & WALS &  \\ 
  Toba & toba1269 & tob & Guaicuruan & SE South America & IDS & Initial & WALS &  \\ 
  Northern Haida & nort2938 & hno & Haida & Alaska-Oregon & IDS &  &  &  \\ 
  White Hmong & hmon1333 & hmd & Hmong-Mien & Southeast Asia & WOLD &  &  &  \\ 
  Romani & vlax1238 & rka & Indo-European & Europe & IDS &  &  &  \\ 
  Judeo-Tat & jude1256 &  & Indo-European & Greater Mesopotamia & IDS &  &  &  \\ 
  Mahasu Pahari  & maha1287 &  & Indo-European & Indic & IDS &  &  &  \\ 
  Tocharian B & tokh1243 &  & Indo-European & Indic & IDS &  &  &  \\ 
  Tocharian A & tokh1242 &  & Indo-European & Inner Asia & IDS &  &  &  \\ 
  Albanian  & tosk1239 & alb & Indo-European & Europe & IDS & Initial & S\&R &  \\ 
  Yiddish & east2295 & ydb & Indo-European & Europe & IDS & Initial & S\&R & \cite[38]{zucker_yiddish1994.pdf} \\ 
  Serbo-Croatian & sout1528 & scr & Indo-European & Greater Mesopotamia & IDS & Initial & S\&R &  \\ 
  Bengali & beng1280 & ben & Indo-European & Indic & IDS & Non-Initial & S\&R & \cite[63...]{hudson_bengali1965.pdf} \\ 
  Breton & bret1244 & bre & Indo-European & Europe & IDS & Initial & WALS &  \\ 
  Bulgarian & bulg1262 & bul & Indo-European & Europe & IDS & Initial & WALS &  \\ 
  Spanish & stan1288 & spa & Indo-European & Europe & IDS & Initial & WALS &  \\ 
  Armenian  & east2283 & arm & Indo-European & Greater Mesopotamia & IDS & Non-Initial & WALS &  \\ 
  Persian & west2369 & prs & Indo-European & Greater Mesopotamia & IDS & Non-Initial & WALS &  \\ 
  Hindi & hind1269 & hin & Indo-European & Indic & IDS & Non-Initial & WALS &  \\ 
  Marathi & mara1378 & mhi & Indo-European & Indic & IDS & Non-Initial & WALS &  \\ 
  Nepali & east1436 & nep & Indo-European & Indic & IDS & Non-Initial & WALS &  \\ 
  Punjabi & west2386 & pan & Indo-European & Indic & IDS & Non-Initial & WALS &  \\ 
  Russian & russ1263 & rus & Indo-European & Inner Asia & IDS & Initial & WALS &  \\ 
  Lower Sorbian & lowe1385 & srl & Indo-European & Europe & WOLD &  &  &  \\ 
  Romanian & roma1327 & mol & Indo-European & Europe & WOLD &  &  &  \\ 
  Selice Romani & west2376 &  & Indo-European & Europe & WOLD &  &  &  \\ 
  Dutch & dutc1256 & dbr & Indo-European & Europe & WOLD & Initial & S\&R & \cite[]{} \\ 
  Old High German & oldh1241 &  & Indo-European & Europe & WOLD & Initial & S\&R & \cite[54]{axel_old-high-german-syntax2007.pdf} \\ 
  Saramaccan & sara1340 & srm & Indo-European & NE South America & WOLD & Initial & S\&R & \cite[30]{byrne_saramaccan1985.pdf} \\ 
  English & stan1293 & eng & Indo-European & Europe & WOLD & Initial & WALS &  \\ 
  Itonama & iton1250 & ito & Itonama & NE South America & IDS & Initial & S\&R & \cite[251,255]{crevels_itonama2012_o} \\ 
  Japanese & nucl1643 & jpn & Japonic & N Coast Asia & WOLD & Non-Initial & WALS &  \\ 
  Aguaruna & agua1253 & agr & Jivaroan & Andean & IDS &  &  &  \\ 
  Karok & karo1304 & krk & Karok & California & IDS &  &  &  \\ 
  Qawasqar & qawa1238 & qaw & Kawesqar & SE South America & IDS &  &  &  \\ 
  Kunza & kunz1244 & atm & Kunza & Andean & IDS &  &  &  \\ 
  Lengua & leng1262 & lng & Lengua-Mascoy & SE South America & IDS &  &  &  \\ 
  Sanapaná  & sana1281 &  & Lengua-Mascoy & SE South America & IDS &  &  &  \\ 
  Chorote & iyoj1235 & crt & Matacoan & SE South America & IDS &  &  &  \\ 
  Maca & maca1260 & mca & Matacoan & SE South America & IDS &  &  &  \\ 
  Nivaclé & niva1238 & chu & Matacoan & SE South America & IDS &  &  &  \\ 
  Wichí & wich1264 & wch & Matacoan & SE South America & WOLD & Initial & WALS &  \\ 
  Q'eqchi' & kekc1242 & kek & Mayan & Mesoamerica & WOLD & Initial & S\&R & \cite[84,96]{stewart_kekchi1980v2.pdf} \\ 
  Zinacantán Tzotzil & tzot1264 & tzz & Mayan & Mesoamerica & WOLD & Initial & S\&R &  \\ 
  Mosetén & mose1249 & mos & Mosetén-Chimané & NE South America & IDS & Initial & WALS &  \\ 
  Movima & movi1243 & mov & Movima & NE South America & IDS & Initial & WALS &  \\ 
  Hup & hupd1244 & hpd & Nadahup & NE South America & WOLD & Initial & WALS &  \\ 
  Aghul & aghu1253 & agl & Nakh-Daghestanian & Greater Mesopotamia & IDS &  &  &  \\ 
  Akhvakh  & akhv1239 & axv & Nakh-Daghestanian & Greater Mesopotamia & IDS &  &  &  \\ 
  Andi & andi1255 & anx & Nakh-Daghestanian & Greater Mesopotamia & IDS &  &  &  \\ 
  Bagvalal & bagv1239 & bgv & Nakh-Daghestanian & Greater Mesopotamia & IDS &  &  &  \\ 
  Botlikh & botl1242 &  & Nakh-Daghestanian & Greater Mesopotamia & IDS &  &  &  \\ 
  Budukh & budu1248 & bdk & Nakh-Daghestanian & Greater Mesopotamia & IDS &  &  &  \\ 
  Chamalal & cham1309 & chm & Nakh-Daghestanian & Greater Mesopotamia & IDS &  &  &  \\ 
  Dargwa & darg1241 & drg & Nakh-Daghestanian & Greater Mesopotamia & IDS &  &  &  \\ 
  Godoberi & ghod1238 & god & Nakh-Daghestanian & Greater Mesopotamia & IDS &  &  &  \\ 
  Hinuq & hinu1240 & hnk & Nakh-Daghestanian & Greater Mesopotamia & IDS &  &  &  \\ 
  Hunzib & hunz1247 & hzb & Nakh-Daghestanian & Greater Mesopotamia & IDS &  &  &  \\ 
  Ingush & ingu1240 & ing & Nakh-Daghestanian & Greater Mesopotamia & IDS &  &  &  \\ 
  Karata & kara1474 & krt & Nakh-Daghestanian & Greater Mesopotamia & IDS &  &  &  \\ 
  Khwarshi  & inxo1238 & khv & Nakh-Daghestanian & Greater Mesopotamia & IDS &  &  &  \\ 
  Kryz & kryt1240 & krz & Nakh-Daghestanian & Greater Mesopotamia & IDS &  &  &  \\ 
  Lak & lakk1252 & lak & Nakh-Daghestanian & Greater Mesopotamia & IDS &  &  &  \\ 
  Rutul & rutu1240 & rut & Nakh-Daghestanian & Greater Mesopotamia & IDS &  &  &  \\ 
  Tabasaran  & sout2752 & tbs & Nakh-Daghestanian & Greater Mesopotamia & IDS &  &  &  \\ 
  Tindi & tind1238 & tnd & Nakh-Daghestanian & Greater Mesopotamia & IDS &  &  &  \\ 
  Tsakhur & tsak1249 & tsa & Nakh-Daghestanian & Greater Mesopotamia & IDS &  &  &  \\ 
  Tsez   & dido1241 & tsz & Nakh-Daghestanian & Greater Mesopotamia & IDS &  &  &  \\ 
  Udi & udii1243 & udi & Nakh-Daghestanian & Greater Mesopotamia & IDS &  &  &  \\ 
  Avar & avar1256 & ava & Nakh-Daghestanian & Greater Mesopotamia & IDS & Non-Initial & WALS &  \\ 
  Chechen & chec1245 & chc & Nakh-Daghestanian & Greater Mesopotamia & IDS & Non-Initial & WALS &  \\ 
  Khinalug & khin1240 & khi & Nakh-Daghestanian & Greater Mesopotamia & IDS & Non-Initial & WALS &  \\ 
  Lezgian & lezg1247 & lez & Nakh-Daghestanian & Greater Mesopotamia & IDS & Non-Initial & WALS &  \\ 
  Tsova-Tush & bats1242 & ttu & Nakh-Daghestanian & Greater Mesopotamia & IDS & Non-Initial & WALS &  \\ 
  Bezhta & bezh1248 & bez & Nakh-Daghestanian & Greater Mesopotamia & WOLD &  &  &  \\ 
  Archi & arch1244 & arc & Nakh-Daghestanian & Greater Mesopotamia & WOLD & Non-Initial & WALS &  \\ 
  Uncunwee & ghul1238 & nbh & Nubian & African Savannah & IDS &  &  &  \\ 
  Kaingáng & kain1272 & kng & Nuclear-Macro-Je & SE South America & IDS & Mixed & S\&R & \cite[163-171]{dasilva_kaingang-paulista2011}\cite[156]{wieseman_kaingang1972.pdf} \\ 
  Chatino  & zaca1242 & cya & Otomanguean & Mesoamerica & IDS & Initial & WALS &  \\ 
  Otomi & mezq1235 & otm & Otomanguean & Mesoamerica & WOLD & Initial & WALS &  \\ 
  Páez & paez1247 & pae & Páez & NE South America & IDS &  &  &  \\ 
  Gurindji & guri1247 & gji & Pama-Nyungan & N Australia & WOLD &  &  &  \\ 
  Catuquina & pano1254 &  & Panoan & NE South America & IDS & Initial & S\&R & \cite[38]{aguiar_katukina1988}\cite[231]{aguiar_katukina1994_o.pdf} \\ 
  Yaminahua & yami1256 & yam & Panoan & NE South America & IDS & Initial & S\&R & \cite[32..]{FaustLoosGramaticaYaminahua.pdf} \\ 
  Chácobo & chac1251 & cbo & Panoan & NE South America & IDS & Initial & WALS &  \\ 
  Shipibo-Conibo & ship1254 & shk & Panoan & NE South America & IDS & Non-Initial & WALS &  \\ 
  Yagua & yagu1244 & yag & Peba-Yagua & NE South America & IDS & Initial & WALS &  \\ 
  Gününa Küne & puel1244 & gku & Puelche & SE South America & IDS &  &  &  \\ 
  Puinave & puin1248 & pui & Puinave & NE South America & IDS & Initial & S\&R & \cite[209]{higuita_puinave2008_s.pdf} \\ 
  Yaruro & pume1238 & yrr & Pumé & NE South America & IDS &  &  &  \\ 
  Imbabura Quechua & imba1240 & qim & Quechuan & Andean & WOLD & Initial & WALS &  \\ 
  Kanuri & cent2050 & knr & Saharan & African Savannah & WOLD & Initial & WALS &  \\ 
  Seri & seri1257 & ser & Seri & Mesoamerica & IDS & Non-Initial & WALS &  \\ 
  Tibetan & tibe1272 & jad & Sino-Tibetan & Inner Asia & IDS &  &  &  \\ 
  Manange & mana1288 & nmm & Sino-Tibetan & Indic & WOLD &  &  &  \\ 
  Mandarin Chinese & mand1415 & kug & Sino-Tibetan & Southeast Asia & WOLD &  &  &  \\ 
  Araona & arao1248 & ana & Tacanan & NE South America & IDS &  &  &  \\ 
  Cavineña & cavi1250 & cav & Tacanan & NE South America & IDS & Initial & S\&R & \cite[102]{guillaume_cavinena2008v3}\cite[599]{15_Pronouns.pdf} \\ 
  Ese Ejja & esee1248 & ese & Tacanan & NE South America & IDS & Initial & WALS &  \\ 
  Tacana & taca1256 & tac & Tacanan & NE South America & IDS & Initial & WALS &  \\ 
  Dehong & tain1252 &  & Tai-Kadai & Southeast Asia & IDS &  &  &  \\ 
  Ecun Buyang & lang1316 &  & Tai-Kadai & Southeast Asia & IDS &  &  &  \\ 
  Hlai  & hlai1239 & lic & Tai-Kadai & Southeast Asia & IDS &  &  &  \\ 
  Khamuang  & nort2740 &  & Tai-Kadai & Southeast Asia & IDS &  &  &  \\ 
  Lakkia & lakk1238 & lkk & Tai-Kadai & Southeast Asia & IDS &  &  &  \\ 
  Langja Buyang & emab1235 &  & Tai-Kadai & Southeast Asia & IDS &  &  &  \\ 
  Nung-Lazhai & guib1244 &  & Tai-Kadai & Southeast Asia & IDS &  &  &  \\ 
  Qau Kelao & qaua1234 &  & Tai-Kadai & Southeast Asia & IDS &  &  &  \\ 
  Sanchong Kelao & gree1278 &  & Tai-Kadai & Southeast Asia & IDS &  &  &  \\ 
  Shan  & shan1277 & sha & Tai-Kadai & Southeast Asia & IDS &  &  &  \\ 
  Southern Tai  & sout2746 &  & Tai-Kadai & Southeast Asia & IDS &  &  &  \\ 
  Sui & suii1243 & sui & Tai-Kadai & Southeast Asia & IDS &  &  &  \\ 
  Thai  & sout2745 &  & Tai-Kadai & Southeast Asia & IDS &  &  &  \\ 
  Zhuang-Longzhou & zuoj1238 &  & Tai-Kadai & Southeast Asia & IDS &  &  &  \\ 
  Maonan & maon1241 &  & Tai-Kadai & Southeast Asia & IDS & Non-Initial & S\&R & \cite[178]{lu_maonan2008} \\ 
  Tai Lü & luuu1242 & lu & Tai-Kadai & Southeast Asia & IDS & Non-Initial & S\&R & \cite[48]{hartmann_tai-lue1984.pdf} \\ 
  Nung-Ninbei & nung1283 & nun & Tai-Kadai & Southeast Asia & IDS & Non-Initial & WALS &  \\ 
  Tai Khün & khun1259 & khn & Tai-Kadai & Southeast Asia & IDS & Non-Initial & WALS &  \\ 
  Thai & thai1261 & tha & Tai-Kadai & Southeast Asia & WOLD & Non-Initial & WALS &  \\ 
  Trumai & trum1247 & tru & Trumai & NE South America & IDS & Initial & WALS &  \\ 
  Tsimshian & nucl1649 & tsi & Tsimshian & Alaska-Oregon & IDS & Non-Initial & WALS &  \\ 
  Siona & sion1247 & sin & Tucanoan & Andean & IDS & Initial & S\&R & \cite[192]{wheeler_siona2000.pdf} \\ 
  Tuyuca & tuyu1244 & tuy & Tucanoan & NE South America & IDS & Initial & S\&R & \cite[443]{barnes-malone_tuyuca2000_o.pdf} \\ 
  Oroqen & oroq1238 &  & Tungusic & Inner Asia & WOLD &  &  &  \\ 
  Aché & ache1246 & ach & Tupian & SE South America & IDS &  &  &  \\ 
  Chiriguano & east2555 & crg & Tupian & SE South America & IDS &  &  &  \\ 
  Sirionó & siri1273 & srn & Tupian & NE South America & IDS & Initial & WALS &  \\ 
  Wayampi & waya1270 & way & Tupian & NE South America & IDS & Initial & WALS &  \\ 
  Guaraní & para1311 & gua & Tupian & SE South America & IDS & Non-Initial & WALS &  \\ 
  Kumyk & kumy1244 & kuq & Turkic & Greater Mesopotamia & IDS &  &  &  \\ 
  Nogai & noga1249 & nog & Turkic & Inner Asia & IDS &  &  &  \\ 
  Azerbaijani & nort2697 & aze & Turkic & Greater Mesopotamia & IDS & Non-Initial & WALS &  \\ 
  Sakha & yaku1245 & ykt & Turkic & Inner Asia & WOLD & Non-Initial & S\&R & \cite[169]{petrova_sakha-yakut2011.pdf} \\ 
  Erzya Mordvin & erzy1239 & moe & Uralic & Inner Asia & IDS &  &  &  \\ 
  Komi & komi1268 & kzy & Uralic & Inner Asia & IDS &  &  &  \\ 
  Mari & east2328 & mme & Uralic & Inner Asia & IDS &  &  &  \\ 
  Nenets & nene1249 & nen & Uralic & Inner Asia & IDS &  &  &  \\ 
  Udmurt & udmu1245 & udm & Uralic & Inner Asia & IDS &  &  &  \\ 
  Northern Saami & nort2671 & sno & Uralic & Inner Asia & IDS & Initial & S\&R & \cite[244]{infoonpitesaami(initial)-wilbur_pite-saami2014} \\ 
  Hungarian & hung1274 & hun & Uralic & Europe & IDS & Non-Initial & WALS &  \\ 
  Estonian & esto1258 & est & Uralic & Inner Asia & IDS & Initial & WALS &  \\ 
  Finnish & finn1318 & fin & Uralic & Inner Asia & IDS & Initial & WALS &  \\ 
  Khanty & khan1273 & kty & Uralic & Inner Asia & IDS & Non-Initial & WALS &  \\ 
  Mansi & mans1258 & mns & Uralic & Inner Asia & IDS & Non-Initial & WALS &  \\ 
  Selkup & selk1253 & skp & Uralic & Inner Asia & IDS & Non-Initial & WALS &  \\ 
  Kildin Saami & kild1236 & ski & Uralic & Inner Asia & WOLD &  &  &  \\ 
  Chipaya & chip1262 & cpy & Uru-Chipaya & Andean & IDS & Initial & S\&R & \cite[65]{cerron-palomino_chipaya2009}\cite[243]{cerron_chipaya2006.pdf} \\ 
  Nahuatl  & high1278 & nsz & Uto-Aztecan & Mesoamerica & IDS &  &  &  \\ 
  Yaqui & yaqu1251 & yaq & Uto-Aztecan & Mesoamerica & WOLD & Non-Initial & WALS &  \\ 
  Nuu-chah-nulth & nuuc1236 & nuu & Wakashan & Alaska-Oregon & IDS &  &  &  \\ 
  Waorani & waor1240 & wao & Waorani & NE South America & IDS &  &  &  \\ 
  Yagán & yama1264 & yah & Yámana & SE South America & IDS & Non-Initial & S\&R & \cite[49]{Yamana(Adam_1885).pdf} \\ 
  Yanomámi & yano1262 & ynm & Yanomam & NE South America & IDS & Initial & S\&R & \cite[73...]{emiri_yanomame1981.pdf}\cite[119..]{lizot_yanomami1996_o.pdf} \\ 
  Ninam  & nina1238 & shi & Yanomam & NE South America & IDS & Initial & WALS &  \\ 
  Ket & kett1243 & ket & Yeniseian & Inner Asia & WOLD & Non-Initial & WALS &  \\ 
  Yuwana & yuwa1244 &  & Yuwana & NE South America & IDS &  &  &  \\ 
  Ayoreo & ayor1240 & ayr & Zamucoan & SE South America & IDS & Non-Initial & S\&R & \cite[393-395]{bertinetto_ayoreo2014_o}\cite[12-Nov]{bertinetto_ayoreo[nd]} \\ 
  Zuni & zuni1245 & zun & Zuni & Basin and Plains & IDS & Initial & WALS &  \\ 
   \hline
\end{longtable}
\end{table}

}

\clearpage
\newpage

\bibliographystyle{apalike}
\bibliography{grammarRefs}


\end{document}